\documentclass[12pt]{article}
\usepackage[utf8]{inputenc}
\usepackage[a4paper,margin=3cm]{geometry}
\usepackage{setspace}\onehalfspacing
\usepackage{parskip}
\usepackage{csquotes}
%\usepackage[backend=biber,style=authoryear]{biblatex}
%\addbibresource{references.bib}
\usepackage{natbib}
\bibliographystyle{apsr}

\title{Introduction to Political Sociology}
\author{Susumu Shikano}
\date{July 2025}

\begin{document}

\maketitle

\section*{What is Political Sociology?}

Political sociology investigates the relationship between politics and society. It addresses three main concerns:
\begin{enumerate}
    \item the  socio-structural conditions of political order and political action,
    \item the structure and function of political institutions and decision-making processes,
    \item and the impact of political decisions and structures on society.
\end{enumerate}
While political science tends to focus on formal institutions and processes, political sociology emphasizes the social foundations of political action and structure.

The following introduces three foundational perspectives in political sociology, each offering a distinct conceptualization of the state and its relationship to society. 

We begin with Max Weber, whose Staatssoziologie develops a definition of the modern state as a political organization that claims the monopoly on the legitimate use of physical force within a given territory. Weber's focus is on the nature of political rules (Herrschaft), the legitimacy of authority, and the role of social action in shaping political order.

The second perspective stems from Talcott Parsons, whose systems theory conceives of society as a functionally differentiated system composed of interrelated subsystems, including the polity. Parsons situates the political system within his AGIL scheme—as the subsystem responsible for goal attainment—and emphasizes its role in ensuring the collective capacity of society to act. 

Finally, we turn to Karl Marx and his political-historical analysis in The Eighteenth Brumaire of Louis Bonaparte. This text not only exemplifies Marx’s materialist conception of history—focusing on class structures and economic foundations—but also introduces the important distinction between a class in itself and a class for itself, emphasizing the political conditions under which social classes become agents of change.


\section*{Classical Foundations}

\subsection*{Max Weber's State Sociology and Political Action}

Max Weber’s sociology of the state begins with a fundamental question: What is the political? In Weber’s understanding, the political refers to the state or to actions carried out by the state. Political action, therefore, is defined as the exercise of state authority or ruling (Herrschaft).

Weber’s sociological approach is grounded in what he calls “verstehende Soziologie,” or interpretive sociology. This is a scientific endeavor aimed at  understanding social interaction through insight of individuals' behavior  and thereby causally explaining its course and consequences. Social action, in this context, is defined as action that is meaningfully oriented toward the behavior of others \citep{weber22}. Weber distinguishes four types of meaningful social action: instrumentally rational action (zweckrational), which is oriented toward the efficient achievement of goals; value-rational action (wertrational), which is guided by belief in the inherent value of the act itself; affectual action, driven by emotions; and traditional action, shaped by internalized habits.

Weber offers a layered conceptualization of the state, beginning with the idea of a social relationship, which may be either open or closed. Among the closed relationship, he introduces the notion of an association (Verband), which is characterized by presence of a leader or administrative stuffs who enforce the internal rule to the members. Weber also distinguishes types of associations by membership. Voluntary associations (Vereine) are based on free choice, whereas institutions (Anstalten) imply compulsory membership, as is the case with the state. A more specific form of the associations is the ruling association (Herrschaftsverband), which includes a ruling structure. In distinguishing forms of control, Weber separates power (Macht) from ruling (Herrschaft). Power is defined as the probability that one actor within a social relationship will be able to impose their will despite resistance. Ruling, by contrast, is the possibility that a command will be obeyed by a given group of people. Crucially, ruling  implies legitimacy and acceptance, whereas power may not. Among these, the political association (politischer Verband) is a type of ruling association oriented toward political leadership. The most specific and institutionalized form is the state, which Weber defines as a political institutional organization whose administrative staff successfully claims the monopoly on the legitimate use of physical force for the enforcement of order. This definition linking the state to a monopoly on legitimate violence is central to Weber’s contribution to modern political sociology. We also have to emphasize this definition is based on the individual behavior.


While political actions are defined as exercise of state authority, those actions which aim to influence the leadership of a political association are considered politically oriented. These include political participation, various actions in elections, or lobbying. Max Weber states that these are the main objective of political sociology, while political science focus more on political actions.

In sum, Weber’s state sociology provides a systematic conceptual framework for understanding the political realm, rooted in his broader theory of social action. The state is understood not simply as an apparatus of power, but as a specific form of political ruling characterized by the legitimate use of physical force and institutionalized administrative structures.



\subsection*{Talcott Parsons: Politics as a Social Subsystem}

In system theory, politics can be understood as an open, goal-oriented behavioral system. This system exists within an environment from which it is distinguishable, though it remains open to environmental influences. Consequently, politics can be conceived as a boundary-maintaining system. Structural and procedural changes within the system are seen as responses to disturbances -— whether originating externally or internally -— reflecting attempts by system members to cope with these disruptions. The political system is thus a self-regulating system capable of maintaining equilibrium. Its capacity to survive disturbances depends on feedback—information or influence relayed back into the system.

Talcott Parsons' conceptualization of social systems builds on this idea, asserting that the two most general functional requirements of human societies are the reproduction of the species and the production of resources -— both of which take place within the broader challenge of antagonistic cooperation. Reproduction corresponds to vertical stratification, while production and problem-solving are associated with horizontal differentiation \citep{parsons1937}.

Parsons conceives of society as a vertically layered action system comprising four interconnected subsystems. The cultural system provides orientation and meaning through values, norms, and symbolic meanings. The social system encompasses roles, institutions, and interactions that regulate behavior via expectations. The personal system refers to individuals’ psychological structures—such as motives and needs—that underlie their capacity for action. Finally, the organismic system represents the physical, biological basis of the individual and guarantees physical existence. These layers are connected: cultural values are institutionalized in the social system; the social system socializes the individual through internalization; and individuals learn through interaction with their biological base. Directionally, control flows top-down, while energy or capacity flows bottom-up.

Horizontally, societies are differentiated into subsystems that fulfill basic functional prerequisites. Parsons identifies four such functions using the AGIL framework: Adaptation (A), referring to resource acquisition; Goal Attainment (G), involving the prioritization of collective goals; Integration (I), referring to the binding together of social units through solidarity; and Latent Pattern Maintenance (L), which refers to the preservation and adaptation of core structures and values over time. Each of these functions is fulfilled by specialized structures—such as roles, organizations, norms, and values. Structural differentiation improves functional performance as roles become more specialized.

In this model, the political system is understood as the organized collective action of society aimed at achieving common goals (G). A society has a differentiated polity if a political association exists. The polity is distinguished from the economy (which is structured by interactive roles in markets) and from the fiduciary system (such as education or religion), which conserves cultural values.

Internally, the polity itself can be analyzed using the AGIL framework. Political institutions are functionally assigned: for example, parliamentary procedures may fulfill integrative functions, while executive organs focus on goal attainment. The political system interacts with the broader societal community through processes of exchange and mutual influence.

Parsons’ theory was heavily influenced by Max Weber, especially in terms of understanding social action as guided by subjective meaning. As in Weber, Parsons distinguishes between goal-rational, value-rational, affectual, and traditional types of action. He also builds on Weber’s idea of societal rationalization—seen in processes like bureaucratization, legal formalization, industrialization, and secularization.

0Parsons’ approach has been subject to criticism, particularly from scholars like  George Homans, who critique it from a critical rationalist perspective \citep{homans1964bringing}. They argue that structural functionalism, in the tradition of Durkheim \citep{durkheim1892division}, treats society as something given and external to individuals, leaving little room for the social construction of society through meaningful individual action. Thus, Parsons' model is seen more as a conceptual and classificatory framework than as a source of explanatory hypotheses.

\subsection*{Karl Marx: Class, the State, and Ideology}

Marxist theory begins with the foundational concept of historical materialism, which asserts that the material conditions of a society. especially its mode of production, shape its social, political, and intellectual life. Central to this view is the distinction between base (Unterbau) and superstructure (Überbau). The base consists of the economic structure of society: the totality of the relations of production, including ownership of the means of production and labor relations. The superstructure, by contrast, encompasses the legal and political institutions as well as cultural forms that arise from and help stabilize the economic base. According to Marx, it is not consciousness that determines being, but rather social being that determines consciousness.

In a capitalist society, the fundamental antagonism is between the bourgeoisie, who own the means of production, and the proletariat, who must sell their labor to survive. These classes are not just economic categories but political agents engaged in a continual class struggle. Marx distinguishes between classes and older forms of social stratification such as estates (Stände), which were defined by law and tradition in feudal society. A class, by contrast, is characterized by its position within the relations of production, for example, by whether one owns capital or sells labor.

Marx introduces an important theoretical refinement in his distinction between a class in itself (Klasse an sich) and a class for itself (Klasse für sich). A class in itself exists objectively: it comprises individuals who share the same position in the economic structure. However, such a class is not yet politically active and influential. It becomes a class for itself when its members become aware of their shared interests, develop a collective identity, and begin to act in political concert. This transformation typically occurs through the experience of conflict or struggle, in which individuals recognize their common condition and the need for collective action.

This conceptual distinction is central to Marx’s analysis in The Eighteenth Brumaire of Louis Bonaparte \citep{Marx1869}, a work in which he reflects on the failed revolution of 1848 and the rise of authoritarian rule in France. Marx analyzes how Louis Bonaparte (Napoleon III) was able to consolidate power following the collapse of the Second Republic. One of the key sociological insights is that social classes do not automatically act in their objective interests. Marx examines the peasant smallholders (Parzellenbauern), who, although they formed a large and objectively distinct social class, were scattered and isolated by their mode of production. They lacked horizontal ties, political organization, and shared consciousness. As a result, they remained a class in itself, incapable of political action on their own behalf. In this vacuum, they turned to Louis Bonaparte as a figure who would represent them, not through democratic means, but as an authoritative and paternalistic ruler. Their support helped enable the coup of December 1851.

Through this case, Marx shows that the state can become relatively autonomous, particularly when no single class is able to assert hegemony. Traditionally, Marx referred to the state as the “executive committee of the bourgeoisie,” but in this instance, the executive branch gained independence from the legislative assembly and from class-based control. Bonaparte was able to rule over a fragmented elite and a politically passive population by centralizing power in the hands of the executive. This autonomization of the state challenges the deterministic reading of Marxism and opens the door for later theorists such as Antonio Gramsci \citep{gramsci1971prison}, Louis Althusser \citep{althusser1977ideologie}, and Nicos Poulantzas \citep{poulantzas1975political} to explore concepts like ideological hegemony, ideological state apparatuses, and the relative autonomy of the state.

Finally, The Eighteenth Brumaire has significant implications for contemporary political sociology. It warns against assuming a straightforward link between economic interests and political representation. Classes may be fragmented, and elites internally divided, creating space for executive aggrandizement, populist takeovers, and democratic backsliding. Marx’s analysis reveals that political authority can become detached from representative institutions, and the state may act in ways that are not reducible to the interests of a particular class, especially when political fragmentation renders collective action difficult.


\section*{Conclusion}
The three perspectives above offer distinct, yet intersecting lenses through which to understand the relationship between politics and society.

Weber provides a clear and influential definition of the modern state as the institutional embodiment of legitimate rules. His emphasis on the role of legitimacy and the interpretive understanding of political action remains central to modern theories of political authority.

Parsons, by contrast, shifts attention to the systemic functions of politics within a broader social order. His framework underscores the role of political institutions in securing collective goals and systemic equilibrium.

Marx, finally, directs our focus to the material foundations of political power, the dynamics of class struggle, and the possibility of state autonomy in times of elite fragmentation and mass political disorganization. 

These approaches illuminate different dimensions of political life in its social context: the legitimacy of ruling grounded in individual action (Weber), the functionality of the political system as a societal subsystem (Parsons), and the conflictual dynamics between class and state (Marx). These dimensions continue to shape and inform contemporary research in political sociology.




\bibliography{literature}

\end{document}
