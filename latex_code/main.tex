\documentclass[12p,a4]{article}
\usepackage{graphicx} % Required for inserting images

\usepackage{url}
\usepackage{hyperref}

\usepackage{setspace}\onehalfspacing

\usepackage{natbib}

\title{Seminar: German Politics and Society}
\author{Susumu  Shikano}
\date{May 2025}

\begin{document}

\maketitle

\section{Introduction}

This seminar provides an overview of the politics and society of contemporary Germany. Designed for students with an interest in political sociology, comparative politics, or German studies, the course explores how Germany’s political system, social structures, and cultural foundations interact to shape public life and policy outcomes.

We begin with foundational perspectives from political sociology to understand the interplay between political systems and society. The seminar then proceeds through thematic blocks, covering political culture, institutional frameworks, social cleavages, patterns of political behavior, and forms of participation beyond elections. We will also examine how interest groups influence policymaking and how public opinion shapes and reflects political decisions.

The latter part of the seminar turns to pressing contemporary challenges that Germany faces, such as immigration, polarization and populism, climate change, and its evolving role within the European Union. Each session is grounded in scholarly literature and encourages critical engagement with empirical findings and theoretical arguments.

Throughout the seminar, participants are expected to read the assigned texts closely, contribute to discussions, and reflect on how the German case informs broader questions in political sociology and comparative politics. In addition to the assigned readings, the instructor will offer supplementary input drawn from further academic literature, particularly recent empirical research. These additions aim to enrich discussions by connecting the core readings to current academic debates and up-to-date findings. The seminar  aims to deepen understanding of both the specificities of the German case and its relevance in the wider European and global context.

\section{Main text}

\begin{itemize}
    \item Grotz, Florian and Wolfgang Schroeder. The Political System of Germany. Cham: Springer International Publishing, 2023.
\end{itemize}

%\clearpage

\section{Schedule and literature}

\subsection*{July 31}

                \begin{description}
                    \item[Session 1]{Introduction to Political Sociology} 
				\begin{itemize}
				    \item Susumu Shikano, Introduction to the Political Sociology of Germany. Unpublished Manuscript. Available at: \url{https://github.com/SuShikano/SeminarPoliticalSociology/Seminar_German_Politics_and_Society.pdf}
				\end{itemize}
				
\item[Session 2] {Political Culture and Social Values}				

                \begin{itemize}
                %\item \citet{Mannewitz2023}
                    %\item \citet{values&activism2015}
                    \item \citet{Pickel02012023}
                \end{itemize}
				
				
\item[Session 3] {Session 3: Political Institutions and Social Structure}				
				\begin{itemize}
                    \item \cite{Schmidt02042016}
				    %\item \citet{Jann2021}
                    %\item \citet{Jannowitz1958}
                    %\item \citet{Segal1967}
				\end{itemize}
				
\item[Session 4] {Social Cleavages and Party Competition}				

    \begin{itemize}
      %  \item Lipset and Rokkan
       \item \citet[Chapter 6]{grotz2023political} 
%\item \citet{Janowitz&Segal1967}
    %    \item \cite{Grande03072023}
        %\item \citet{Saalfeld2002}
     %   \item \citet{Wagner02012023}
    \end{itemize}


                \end{description}

\subsection*{August 1}

\begin{description}
\item[Session 1] {Voting Behavior}

    \begin{itemize}
         \item \citet[Chapter 5]{grotz2023political} 
        %\item \citet{hudde2023gender}
        %\item \citet{wenner2023voting}
    \end{itemize}
				
				
\item[Session 2] {Political Participation beyond Elections}
				
\begin{itemize}
    %\item 
%\citet{Daphi03072023}
%\item \citet{Weisskircher03072023}
\item \citet{müller-rommel1985}
%\item \citet{Fadaee2021}
\end{itemize}
				
\item[Session 3] {Interest Groups and Policy Making Process}				
				
		\begin{itemize}
            \item \citet[Chapter 7]{grotz2023political} 
		   % \item \citet{Klüver03042015}  
           %\item \citet{Flöthe02042020}
		\end{itemize}
				
\item[Session 4] {Public Opinion and its impact}				
				
\begin{itemize}
    %\item \citet{Noelle-Neumann2006}
    \item \citet{Metag2016}
   % \item \citet{schulz2024}
\end{itemize}

\end{description}				


\subsection*{August 4}

\begin{description}
\item[Session 1] {Political Communication}
                \begin{itemize}
				     \item \citet[Chapter 8]{grotz2023political} 
%\item \citet{Datts01102024}
%\item \citet{Metag2016}
				\end{itemize}
				
\item[Session 2] {Germany and the European Union}				
				
				\begin{itemize}
                   \item \citet[Chapter 3]{grotz2023political} 
				    %\item \citet{Wonka&Haunss2020}
				\end{itemize}
				
\item[Session 3] {Contemporary Challenges: Immigration}
				
\begin{itemize}
    %\item \citet{Hooghe2018}
    %\item \citet{gundacker2025regional}
    \item \citet{hager2019attitudes}
\end{itemize}				
				
 \item[Session 4] {Contemporary Challenges: Polarization and Populism}
				
\begin{itemize}
    %\item \citet{Harteveld2023}
    %\item \citet{Hansen2024}
\item \citet{Arzheimer2015}
\end{itemize}				

\end{description}                

				
\subsection*{August 5}				
\begin{description}
    \item[Session 1]{Contemporary Challenges: Climate Change} 
	\begin{itemize}
	    \item \citet{McCright03032016}
%\item \citet{Mewes2024}
%\item \citet{Rudolph02012018}
	\end{itemize}
 \item[Session 2] {Concluding session}				
\end{description}				
				

\bibliographystyle{apsr}
\bibliography{literature}


\end{document}
