\documentclass[12pt]{article}
\usepackage[utf8]{inputenc}
\usepackage[a4paper,margin=2.5cm]{geometry}
\usepackage{parskip}
\usepackage{csquotes}
\usepackage[backend=biber,style=authoryear]{biblatex}
\addbibresource{references.bib}

\title{Introduction to Political Sociology}
\author{Susumu Shikano}
\date{\today}

\begin{document}

\maketitle

\section*{What is Political Sociology?}

Political sociology investigates the relationship between politics and society. It addresses three main concerns:
\begin{enumerate}
    \item the societal and socio-structural conditions of political order and political action,
    \item the structure and function of political institutions and decision-making processes,
    \item and the impact of political decisions and structures on society.
\end{enumerate}
While political science tends to focus on formal institutions and processes, political sociology emphasizes the social foundations of political action and structure \parencite{pappi2003}.

\section*{Classical Foundations}

\subsection*{Max Weber: State Sociology and Political Action}

Max Weber defined the state as a “political institution that successfully claims the monopoly on the legitimate use of physical force within a given territory” \parencite[§17]{weber1922}. This conception stands at the core of his sociology of the state (\textit{Staatssoziologie}), which forms one of the foundational pillars of political sociology.

Weber distinguished between two key forms of action:
\begin{itemize}
    \item \textbf{Political action} refers to actions carried out by political organizations—especially the legitimate administrative use of authority.
    \item \textbf{Politically oriented action} includes any social action—individual or collective—that aims to influence the leadership or policies of political associations \parencite{weber1976}.
\end{itemize}
It is politically oriented action that forms the core subject matter of political sociology, as it captures how social actors engage with political power beyond formal roles.

\subsection*{Talcott Parsons: Politics as a Social Subsystem}

Talcott Parsons approached political sociology through the lens of structural functionalism. In his general theory of action systems, politics is one of four functionally differentiated subsystems of society, tasked primarily with \textit{goal attainment}—the formulation and implementation of collective objectives \parencite{parsons1959}.

Parsons’ AGIL model identifies four systemic imperatives:
\begin{itemize}
    \item \textbf{Adaptation} (economy): securing resources,
    \item \textbf{Goal attainment} (politics): pursuing collective aims,
    \item \textbf{Integration} (law, community): ensuring cohesion,
    \item \textbf{Latent pattern maintenance} (culture): preserving values.
\end{itemize}
The political system is therefore seen as a stabilizing force, providing direction and coordination in pursuit of societal goals.

\subsection*{Karl Marx: Class, the State, and Ideology}

Karl Marx viewed the state not as a neutral entity but as “the executive committee for managing the common affairs of the bourgeoisie” \parencite{marx1869}. Political sociology for Marx begins with the material conditions of society—especially the \textit{relations of production}—which determine class structure and political outcomes.

Marx distinguishes between:
\begin{itemize}
    \item \textbf{Class in itself}—a group sharing the same objective position in the production process,
    \item \textbf{Class for itself}—a group that has become politically self-aware and organized.
\end{itemize}
In his study \textit{The Eighteenth Brumaire of Louis Bonaparte}, Marx explores how a fragmented bourgeoisie and weak working class allowed the executive branch to concentrate power. The apparent autonomy of the state, he argues, reflects class dynamics rather than independence \parencite{poulantzas1976}.

\section*{Conclusion}

The classical contributions to political sociology outline distinct traditions:
\begin{itemize}
    \item \textbf{Weber} explicitly founded the states based on individual social actions. In this process, he also make it clear what the objective of political sociology is.
    \item \textbf{Parsons} conceptualizes politics as a subsystem responsible for coordinating societal goals.
    \item \textbf{Marx} focuses on class conflict, state power, and ideological control.
\end{itemize}

These frameworks remain vital for understanding political power, legitimacy, and social change.

\printbibliography

\end{document}
